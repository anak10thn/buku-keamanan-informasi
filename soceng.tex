\chapter{Social Engineering}
Istilah ``social engineering'' dapat bermakna banyak hal. 
Dalam ilmu yang berbeda, maknanya berbeda.
Dalam konteks keamanan informasi, {\em social engineering} adalah
memanipulasi seseorang secara psikologis agar melakukan sebuah aksi
(kegiatan) tertentu atau membocorkan informasi.
Contoh informasi yang dimaksudkan misalnya adalah kata kunci atau
password kita.

{\em Social engineering} umumnya memanfaatkan kelemahan manusia.
Sebagai contoh, kita diajak {\em ngobrol} dan tanpa kita sadari
kita menceritakan tentang nama dan password (atau nomor PIN dari ATM) kita.
Kelemahan manusia dapat muncul jika kita puji-puji yang bersangkutan
atau kita buat merasa nyaman untuk menceritakan hal-hal yang sifatnya privat.


Beberapa contoh dari {\em social engineering} antara lain adalah
hal-hal di bawah ini.

\begin{itemize}
    \item Mengirimkan email palsu {\em phising}, yang mengatakan bahwa
    sistem mengalami perubahan sehingga Anda diharapkan menggantikan
    password atau melakukan konfirmasi ulang mengenai passwordnya.
    Target kemudian diarahkan kepada sebuah situs web gadungan
    untuk memberikan nama (userid) dan password. Penjahat kemudian
    mendapatkan userid dan password dari target.
    \item Pola yang mirip dengan yang di atas, mengirimkan email palsu,
    tetapi pengguna diminta untuk mengklik sebuah tautan tertentu.
    Tautan ini sebetulnya menyisipkan sesuatu (misal malware) ke dalam
    web browser kita sehingga sistem kita kemudian dapat dikendalikan dari
    jarak jauh.
    \item Target ditelepon yang mengatasnamakan sebuah layanan
    (bank, layanan ecommerce, jasa angkutan online, dan sejenisnya)
    dan mengatakan bahwa target memenangkan sebuah undian.
    Untuk memastikan bahwa undian diklaim, target diminta untuk 
    menyebutkan PIN (atau OTP, {\em one time password}, yang berisi nomor)
    yang dikirimkan melalui SMS atau WhatsApp (WA). 
    Jika target menurut, maka penyerang akan masuk ke layanan yang bersangkutan 
    dan menggunakan data dari SMS/WA itu untuk melakukan password reset
    (sesuai dengan password baru yang akan dia berikan).
    Penyerang kemudian akan mendapatkan akun kita dan menghabisi
    uang kita yang disimpan pada layanan tersebut.
    \item {\em Shouldering.} Ini adalah mengintip melalui bahu
    seseorang. Misalnya, penyerang ingin mengetahui password atau
    PIN yang akan diketikkan seseorang, maka dia akan berdiri di
    belakang target dan akan memperhatikan apa yang diketikkan oleh
    target.
\end{itemize}

Salah satu contoh kasus {\em social engineering} yang cukup terkenal
adalah kasus Kevin Mitnick. Dia adalah salah satu contoh pelaku
yang berhasil mengelabui penegak hukum di Amerika sehingga diburu
berkali-kali. Dia memiliki keahlian mengajak orang berbicara (via telepon)
untuk kemudian memberitahukan rahasia dari sistem telepon yang digunakan
oleh penegak hukum di Amerika.
Pada akhirnya Kevin Mitnick berhasil ditangkap dan dimasukkan ke penjara.
Cerita mengenai hal ini dibukukan oleh Tsutomo Simomura dalam bukunya
yang berjudul ``Takedown''~\cite{takedown}. (Cerita ini juga akhirnya
dijadikan film dengan judul yang sama.)
Setelah keluar dari penjara, Kevin Mitnick menjadi konsultan
{\em security} dan menerbitkan beberapa buku tentang {\em social engineering}.
Salah satu bukunya dalah ``The Art of Deception''~\cite{mitnickdeception}.

{\em Social engineering} dapat digunakan sebagai salah satu cara
untuk mengetahui kebocoran di sebuah perusahaan (instansi, institusi).
Meskipun demikain, metodologi untuk melakukannya masih belum standar.